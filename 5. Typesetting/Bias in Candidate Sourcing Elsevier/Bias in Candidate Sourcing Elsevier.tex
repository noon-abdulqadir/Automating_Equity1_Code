% Bias in Candidate Sourcing
\documentclass[preprint, review, 12pt]{elsarticle}

% Elsevier Packages
\usepackage{natbib}
\bibliographystyle{apalike.bst}

% Other Packages
\usepackage{orcidlink}
\usepackage{geometry}
\usepackage{graphicx}
\usepackage{txfonts}
\usepackage{hyperref}
\usepackage{hypernat}
\usepackage{url}
\usepackage{enumitem}
\usepackage{xspace}
\usepackage{float}
\usepackage{rotating, lscape, nicematrix, longtable, tabu, booktabs, siunitx, multirow}
\usepackage{array}
\usepackage{setspace}
\usepackage[utf8]{inputenc}
\usepackage{textgreek}
\usepackage{endnotes}
\usepackage[font=normal, skip=0pt]{caption}
\usepackage{subcaption}

% Tables and Figures path
\graphicspath{{./tables_and_figs/figs/}}

% Make new column types
% Top align
\newcolumntype{C}[1]{>{\centering\let\newline\\\arraybackslash\setstretch{1.0}}p{#1\columnwidth}}
\newcolumntype{L}[1]{>{\raggedright\let\newline\\\arraybackslash\setstretch{1.0}}p{#1\columnwidth}}
\newcolumntype{R}[1]{>{\raggedleft\let\newline\\\arraybackslash\setstretch{1.0}}p{#1\columnwidth}}
% Bottom align
\newcolumntype{S}[1]{>{\centering\let\newline\\\arraybackslash\setstretch{1.0}}b{#1\columnwidth}}
\newcolumntype{A}[1]{>{\raggedright\let\newline\\\arraybackslash\setstretch{1.0}}b{#1\columnwidth}}
\newcolumntype{D}[1]{>{\raggedleft\let\newline\\\arraybackslash\setstretch{1.0}}b{#1\columnwidth}}

% Journal name
\journal{Public Relations Review}

% Body
\begin{document}

\begin{frontmatter}
\title{Bias in Candidate Sourcing Communication:\\Investigating Stereotypical Gender- And Age-Related Frames in Online Job Advertisements at the Sectoral Level\tnoteref{title}}

\author[uva]{Noon M.F. Abdulqadir\corref{cor1} \orcidlink{0000-0002-6243-9241} \fnref{noon}}
\ead{noon.abdulqadir@uva.nl}
\cortext[cor1]{Correspondence concerning this article should be addressed to Noon Abdulqadir, Postbus 15791, 1001 NG Amsterdam. Email: noon.abdulqadir@uva.nl}

\author[uva]{Anne Kroon \orcidlink{0000-0001-7600-7979}}
\author[erasmus]{Martine van Selm \orcidlink{0000-0001-9188-4021}}
\author[uva]{Margot van der Goot \orcidlink{0000-0001-6904-6515}}
\author[wur]{Rens Vliegenthart \orcidlink{0000-0003-2401-2914}}

\address[uva]{Amsterdam School of Communication Research (ASCoR)}
\address[erasmus]{Erasmus School of History, Culture and Communication (ESHCC)}
\address[wur]{Wageningen University and Research (WUR)}

\begin{abstract}
    Studies show the extent to which job advertisements contain stereotypical wordings correlates with the level of segregation at individual job level and boarder occupation level. However, limited research links the framing of job ads to gender- and age-based segregation at the sector level. Guided by the stereotype content model, we operationalize stereotypical warmth and competence-related frames in candidate sourcing communication and investigate their presence in job ads from occupational sectors with varying levels of segregation. Automated content analysis was conducted on a dataset of online job ad sentences (\textit{n}=307154). Results indicate warmth-related frames are most observed in ads from female-dominated (vs. male-dominated) and younger-dominated (vs. older-dominated and mixed-age) dominated sectors. Conversely, competence-related frames are most observed in ads from male-dominated (vs. female-dominated and mixed-gender) and older-dominated (vs. younger-dominated and mixed-age) sectors. Taken together, we present an operationalization of stereotypical warmth- and competence-related frames in early employer communication and posit that social categorization framing may be at play.
    \end{abstract}

\begin{keyword}
    automated content analysis \sep occupational segregation \sep social categorization frames \sep stereotype content model
    \end{keyword}

\end{frontmatter}

Scholars have long documented the influence of interpersonal bias in candidate recruitment, particularly with regard to job seekers’ gender and age \citep{beattie_possible_2012, heilman_gender_2012, paleari_when_2019}. This influence is observable during active candidate sourcing where job advertisements represent the first touchpoint in employer communication \citep{RynesS.1989}. These job ads can signal essential value-related information about an organization \citep{de_cooman_portraying_2012} but can also determine the pool of potential candidates who apply for an advertised position and reinforce existing interpersonal biases. On the micro-level, job ads and the HR decisions that dictate their content are informed by the type of candidate employers explicitly or implicitly envision as “ideal” for a position \citep{kelly_gendered_2010}. \citet{van_selm_search_2021}, for instance, found that job ads targeting older workers contained frames consistent with general stereotypes of older individuals. On the meso-level, the content of job ads can also reflect the level of segregation (homogeneity), or lack thereof (heterogeneity), in an occupation. As \citet{gaucher_evidence_2011} found, job ads from traditionally male-dominated occupations such as engineer, plumber, and security guard tend to contain terms such as competitive, leader, ambitious, and similar wordings that are culturally related to masculinity, consequently making such ads less appealing to female candidates.

\bibliography{study1.bib}

\end{document}
