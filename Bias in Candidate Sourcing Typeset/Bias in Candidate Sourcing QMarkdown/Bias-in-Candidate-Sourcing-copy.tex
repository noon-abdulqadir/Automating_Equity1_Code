% Options for packages loaded elsewhere
\PassOptionsToPackage{unicode}{hyperref}
\PassOptionsToPackage{hyphens}{url}
\PassOptionsToPackage{dvipsnames,svgnames,x11names}{xcolor}
%
\documentclass[
  12pt,
  letterpaper,
  DIV=11,
  numbers=noendperiod]{scrartcl}

\usepackage{amsmath,amssymb}
\usepackage{lmodern}
\usepackage{iftex}
\ifPDFTeX
  \usepackage[T1]{fontenc}
  \usepackage[utf8]{inputenc}
  \usepackage{textcomp} % provide euro and other symbols
\else % if luatex or xetex
  \usepackage{unicode-math}
  \defaultfontfeatures{Scale=MatchLowercase}
  \defaultfontfeatures[\rmfamily]{Ligatures=TeX,Scale=1}
  \setmainfont[]{Times New Roman}
\fi
% Use upquote if available, for straight quotes in verbatim environments
\IfFileExists{upquote.sty}{\usepackage{upquote}}{}
\IfFileExists{microtype.sty}{% use microtype if available
  \usepackage[]{microtype}
  \UseMicrotypeSet[protrusion]{basicmath} % disable protrusion for tt fonts
}{}
\usepackage{xcolor}
\setlength{\emergencystretch}{3em} % prevent overfull lines
\setcounter{secnumdepth}{-\maxdimen} % remove section numbering
% Make \paragraph and \subparagraph free-standing
\ifx\paragraph\undefined\else
  \let\oldparagraph\paragraph
  \renewcommand{\paragraph}[1]{\oldparagraph{#1}\mbox{}}
\fi
\ifx\subparagraph\undefined\else
  \let\oldsubparagraph\subparagraph
  \renewcommand{\subparagraph}[1]{\oldsubparagraph{#1}\mbox{}}
\fi


\providecommand{\tightlist}{%
  \setlength{\itemsep}{0pt}\setlength{\parskip}{0pt}}\usepackage{longtable,booktabs,array}
\usepackage{calc} % for calculating minipage widths
% Correct order of tables after \paragraph or \subparagraph
\usepackage{etoolbox}
\makeatletter
\patchcmd\longtable{\par}{\if@noskipsec\mbox{}\fi\par}{}{}
\makeatother
% Allow footnotes in longtable head/foot
\IfFileExists{footnotehyper.sty}{\usepackage{footnotehyper}}{\usepackage{footnote}}
\makesavenoteenv{longtable}
\usepackage{graphicx}
\makeatletter
\def\maxwidth{\ifdim\Gin@nat@width>\linewidth\linewidth\else\Gin@nat@width\fi}
\def\maxheight{\ifdim\Gin@nat@height>\textheight\textheight\else\Gin@nat@height\fi}
\makeatother
% Scale images if necessary, so that they will not overflow the page
% margins by default, and it is still possible to overwrite the defaults
% using explicit options in \includegraphics[width, height, ...]{}
\setkeys{Gin}{width=\maxwidth,height=\maxheight,keepaspectratio}
% Set default figure placement to htbp
\makeatletter
\def\fps@figure{htbp}
\makeatother
\newlength{\cslhangindent}
\setlength{\cslhangindent}{1.5em}
\newlength{\csllabelwidth}
\setlength{\csllabelwidth}{3em}
\newlength{\cslentryspacingunit} % times entry-spacing
\setlength{\cslentryspacingunit}{\parskip}
\newenvironment{CSLReferences}[2] % #1 hanging-ident, #2 entry spacing
 {% don't indent paragraphs
  \setlength{\parindent}{0pt}
  % turn on hanging indent if param 1 is 1
  \ifodd #1
  \let\oldpar\par
  \def\par{\hangindent=\cslhangindent\oldpar}
  \fi
  % set entry spacing
  \setlength{\parskip}{#2\cslentryspacingunit}
 }%
 {}
\usepackage{calc}
\newcommand{\CSLBlock}[1]{#1\hfill\break}
\newcommand{\CSLLeftMargin}[1]{\parbox[t]{\csllabelwidth}{#1}}
\newcommand{\CSLRightInline}[1]{\parbox[t]{\linewidth - \csllabelwidth}{#1}\break}
\newcommand{\CSLIndent}[1]{\hspace{\cslhangindent}#1}

\KOMAoption{captions}{tableheading,figureheading}
\makeatletter
\makeatother
\makeatletter
\makeatother
\makeatletter
\@ifpackageloaded{caption}{}{\usepackage{caption}}
\AtBeginDocument{%
\ifdefined\contentsname
  \renewcommand*\contentsname{Table of contents}
\else
  \newcommand\contentsname{Table of contents}
\fi
\ifdefined\listfigurename
  \renewcommand*\listfigurename{List of Figures}
\else
  \newcommand\listfigurename{List of Figures}
\fi
\ifdefined\listtablename
  \renewcommand*\listtablename{List of Tables}
\else
  \newcommand\listtablename{List of Tables}
\fi
\ifdefined\figurename
  \renewcommand*\figurename{Figure}
\else
  \newcommand\figurename{Figure}
\fi
\ifdefined\tablename
  \renewcommand*\tablename{Table}
\else
  \newcommand\tablename{Table}
\fi
}
\@ifpackageloaded{float}{}{\usepackage{float}}
\floatstyle{ruled}
\@ifundefined{c@chapter}{\newfloat{codelisting}{h}{lop}}{\newfloat{codelisting}{h}{lop}[chapter]}
\floatname{codelisting}{Listing}
\newcommand*\listoflistings{\listof{codelisting}{List of Listings}}
\makeatother
\makeatletter
\@ifpackageloaded{caption}{}{\usepackage{caption}}
\@ifpackageloaded{subcaption}{}{\usepackage{subcaption}}
\makeatother
\makeatletter
\@ifpackageloaded{tcolorbox}{}{\usepackage[many]{tcolorbox}}
\makeatother
\makeatletter
\@ifundefined{shadecolor}{\definecolor{shadecolor}{rgb}{.97, .97, .97}}
\makeatother
\makeatletter
\makeatother
\ifLuaTeX
\usepackage[bidi=basic]{babel}
\else
\usepackage[bidi=default]{babel}
\fi
\babelprovide[main,import]{american}
% get rid of language-specific shorthands (see #6817):
\let\LanguageShortHands\languageshorthands
\def\languageshorthands#1{}
\ifLuaTeX
  \usepackage{selnolig}  % disable illegal ligatures
\fi
\IfFileExists{bookmark.sty}{\usepackage{bookmark}}{\usepackage{hyperref}}
\IfFileExists{xurl.sty}{\usepackage{xurl}}{} % add URL line breaks if available
\urlstyle{same} % disable monospaced font for URLs
\hypersetup{
  pdftitle={Bias in Candidate Sourcing Communication},
  pdflang={en-US},
  pdfkeywords={Stereotype content model, social categorization
frames, horizontal occupational segregation, supervised content
analysis},
  colorlinks=true,
  linkcolor={blue},
  filecolor={Maroon},
  citecolor={Blue},
  urlcolor={Blue},
  pdfcreator={LaTeX via pandoc}}

\title{Bias in Candidate Sourcing Communication}
\usepackage{etoolbox}
\makeatletter
\providecommand{\subtitle}[1]{% add subtitle to \maketitle
  \apptocmd{\@title}{\par {\large #1 \par}}{}{}
}
\makeatother
\subtitle{Investigating Stereotypical Gender- And Age-Related Frames in
Online Job Advertisements at the Sectoral Level}
\author{true}
\date{}

\begin{document}
\maketitle
\begin{abstract}
Studies show that the extent to which job advertisements contain
stereotypical wordings correlates with the level of segregation in an
occupational domain. However, limited research links the framing of job
ads to social category-based occupational segregation at the macro
sectoral level. Guided by the stereotype content model, the present
study operationalizes stereotypical social categorization frames in
candidate sourcing communication and investigates their presence in job
ads from occupational sectors with varying gender and age segregation.
An automated (supervised) content analysis was conducted on a dataset of
online job ads (\emph{n}=17,050). Results indicate warmth-related frames
are most observed in ads from female-dominated (vs.~male-dominated and
mixed-gender) sectors. Conversely, competence-related frames are most
observed in ads from male-dominated (vs.~female-dominated) and younger
worker-dominated (vs.~older worker-dominated and mixed-age) sectors.
Taken together, we present an operationalization of stereotypical
warmth- and competence-related frames in early employer communication
and posit that social categorization framing may be at play.
\end{abstract}
\ifdefined\Shaded\renewenvironment{Shaded}{\begin{tcolorbox}[borderline west={3pt}{0pt}{shadecolor}, boxrule=0pt, frame hidden, interior hidden, breakable, sharp corners, enhanced]}{\end{tcolorbox}}\fi

Scholars have long documented the influence of interpersonal bias in
candidate recruitment, particularly with regard to job seekers' gender
and age
(\protect\hyperlink{ref-beattie2012PossibleUnconsciousBias}{Beattie \&
Johnson, 2012};
\protect\hyperlink{ref-heilman2012GenderStereotypesWorkplace}{Heilman,
2012}; \protect\hyperlink{ref-paleari2019WhenPrejudiceYou}{Paleari et
al., 2019}). This influence is observable during active candidate
sourcing where job advertisements represent the first touchpoint in
employer communication (\protect\hyperlink{ref-RynesS.1989}{Rynes,
1989}). These job ads can signal essential value-related information
about an organization
(\protect\hyperlink{ref-decooman2012PortrayingFittingValues}{De Cooman
\& Pepermans, 2012}) but can also determine the pool of potential
candidates who apply for an advertised position and reinforce existing
interpersonal biases. On the micro-level, job ads and the HR decisions
that dictate their content are informed by the type of candidate
employers explicitly or implicitly envision as ``ideal'' for a position
(\protect\hyperlink{ref-kelly2010GenderedChallengeGendered}{Kelly et
al., 2010}). van Selm \& van den Heijkant
(\protect\hyperlink{ref-vanselm2021SearchOlderWorker}{2021}), for
instance, found that job ads targeting older workers contained frames
consistent with general stereotypes of older individuals. On the
meso-level, the content of job ads can also reflect the level of
segregation (homogeneity), or lack thereof (heterogeneity), in an
occupation. Gaucher et al.
(\protect\hyperlink{ref-gaucher2011EvidenceThatGendered}{2011}) found,
job ads from traditionally male-dominated occupations such as engineer,
plumber, and security guard tend to contain terms such as competitive,
leader, ambitious, and similar wordings that are culturally related to
masculinity, consequently making such ads less appealing to female
candidates.

Thus far, limited research has explored the relationship between the
content of job advertisements and macro-level occupational segregation,
i.e., at the sectoral level, demarcated specifically by social category
composition (with the exception of
\protect\hyperlink{ref-garcia-retamero2006PrejudiceWomenMalecongenial}{Garcia-Retamero
\& López-Zafra, 2006}; as cited in
\protect\hyperlink{ref-clarke2020GenderStereotypesGenderTyped}{Clarke,
2020}). Although career changes across sectors are more common than ones
across occupations
(\protect\hyperlink{ref-carrillo-tudela_extent_2016}{Carrillo-Tudela et
al., 2016}), studies show workers generally tend to stay within social
category-typed job domains due to a backlash effect
(\protect\hyperlink{ref-fritsch_horizontal_2020}{Fritsch et al., 2020}).
Sectoral segregation demarcated based on social category composition may
thus be more persistent and present a higher social barrier to entry
than segregation demarcated based on other occupational factors, thus
becoming key to understanding possible inequity in candidate sourcing.

Given this, the present study makes three main contributions. First, it
investigates candidate sourcing in Dutch intranational sectors that are
heterogeneous and homogenous in gender and age composition, i.e.,
female- and male-dominated vs.~mixed-gender sectors and older worker-
and younger worker-dominated vs.~mixed-age sectors respectively (see
Table 10). We take a communication perspective and specifically examine
the extent to which framed gender- and age-related stereotypes are
present in online job ads from heterogeneous and homogenous sectors.
Second, it presents a systematic operationalization of broad-level
gender- and age-related stereotypical frames in job ads. Employing the
social categorization framing hypothesis
(\protect\hyperlink{ref-Yang2015a}{Yang, 2015}), we adopt the
conceptualization of pancultural and generalizable warmth- and
competence-related social category stereotypes put forth in the
stereotype content model {[}SCM; Fiske et al.
(\protect\hyperlink{ref-fiske2002ModelOftenMixed}{2002}){]}. Third, we
utilize a rigorous automated content analysis method that leverages
holistic singular assessment coding and word vector representations as
embeddings for supervised binary classification
(\protect\hyperlink{ref-carducci_twitpersonality_2018}{Carducci et al.,
2018}). This study thus seeks to answer the question: \emph{to what
extent are job advertisements from different occupational sectors framed
in terms of warmth- and competence-based gender and age stereotypes?}

\hypertarget{framing}{%
\subsection{Framing Theory and Stereotypical Frames}\label{framing}}

Framing a message constrains its audiences to desired and meaningful
interpretations by directing attention to information judged to be
important by the message sender. Frames make salient some aspects or
subset of possible considerations about a subject over others
(\protect\hyperlink{ref-entman1993FramingClarificationFractured}{Entman,
1993}), typically through strategic ``selection, emphasis, exclusion,
and elaboration''
(\protect\hyperlink{ref-reese2001FramingPublicLife}{Reese et al., 2001,
p. 10}). Within framing theory, stereotypes are a powerful framing
device underscored by culturally embedded implicit reasoning devices
(\protect\hyperlink{ref-vangorp2005WhereFrameVictims}{Van Gorp, 2005}),
i.e., they draw on and activate culturally shared (consensual) cognitive
schemata. In their capacity as framing devices, stereotypes draw
attention to a particular assessment of social categories, their roles,
and their distance from the reader, thus stereotypes may come to define
a frame.

Expounding on the use of stereotypes in framing,
(\protect\hyperlink{ref-Yang2015a}{Yang, 2015}) presents a typology of
stereotypical frame genres differentiated through their effects on
individual cognition, and the pathway by which they make salient the
perceived social distance between categories, i.e., degree of emphasis
on the self-to-other differences. \emph{Social categorization} frames in
particular are germane to the current study as their usage centers
around ownership of cultural objects such as social roles or certain
jobs and occupational sectors. By emphasizing the belongingness of
cultural objects to select social categories, social categorization
frames activate distinct social identities, otherization, and make
salient the social distance between social categories. This frame genre
thus conveys the reasoning that ``certain groups are outgroups and their
members are not qualified for ingroup activities''
(\protect\hyperlink{ref-Yang2015a}{Yang, 2015, p. 261}). Likewise,
social categorization frames may activate self-stereotyping and lead to
ingroup members assuming the characteristics stereotypically associated
with their category, increasing conformity and deindividuation
(\protect\hyperlink{ref-brown2003BlackwellHandbookSocial}{Brown \&
Gaertner, 2003}).

Social categorization frames are also applied differently to different
categories depending on whether they are dominant or non-dominant in a
domain. When addressing dominant social categories, emphasis is placed
on ingroup characteristics and their complementarity to features of the
cultural object. When addressing outgroups, the information also tends
to be stereotype-consistent, however, the emphasis is on the mismatch
between the cultural object and the categories' characteristics.
Examples of social categorization framing include female political
candidates being framed as intruders and a novelty in political races by
the news media (\protect\hyperlink{ref-meeks_all_2013}{Meeks, 2013};
\protect\hyperlink{ref-sullivan19891984VicePresidential}{Sullivan,
1989}) and female athletes being depicted as aliens and given minor
roles in the sports coverage
(\protect\hyperlink{ref-hardin2002FramingSexualDifference}{Hardin et
al., 2002}). Applied to job ads wherein messages are targeted to
perceived ideal candidates, frames in job ads from sectors that are
``owned'' by a single social category, i.e., from a homogeneous sector,
are likely to emphasize characteristics perceived as essential to the
dominant social category.

\hypertarget{scm}{%
\subsection{Stereotype Content Model (SCM)}\label{scm}}

Examining consensual stereotypes, i.e., stereotypes that are (perceived
to be) shared by the wider culture
(\protect\hyperlink{ref-Zanna2013}{Gardner, 1994}), Beukeboom \& Burgers
(\protect\hyperlink{ref-beukeboom2019HowStereotypesAre}{2019}) describe
stereotype content as the ``{[}cognitive{]} representation people hold
about a social category, consisting of beliefs and expectancies about
probable behaviors, features, and traits'' (p.~9). The extent to which
stereotype content is endorsed depends on the strength of individual
essentialist beliefs about the stereotyped category
(\protect\hyperlink{ref-bastian2006PsychologicalEssentialismStereotype}{Bastian
\& Haslam, 2006}; for perceived category essentialism and stereotyping,
see \protect\hyperlink{ref-beukeboom2019HowStereotypesAre}{Beukeboom \&
Burgers, 2019}).

Specific stereotypes about both gender and age categories can vary
across cultures and within different strata of the same culture. One
model that circumvents this barrier is the \emph{stereotype content
model} (SCM) as it is suitable for investigating pancultural,
superordinate, and broad-level stereotypes. Developed by Fiske et al.
(\protect\hyperlink{ref-fiske2002ModelOftenMixed}{2002}), the SCM
provides a universal principle determining predictors of stereotypes and
sets up a framework to comparatively and systematically investigate
stereotype content
(\protect\hyperlink{ref-kroon2018ReliableUnproductiveStereotypes}{Kroon
et al., 2018}; \protect\hyperlink{ref-vanselm2021SearchOlderWorker}{van
Selm \& van den Heijkant, 2021}). Due to its generalizability and
intuitiveness, the SCM has been routinely used by scholars to analyze
intergroup communication, media, and text for markers of other- as well
as self-stereotyping
(\protect\hyperlink{ref-westerhof2010FillingMissingLink}{Westerhof et
al., 2010}; \protect\hyperlink{ref-white2009ThinkWomenThink}{White \&
Gardner, 2009}). In recruitment, Hofhuis et al.
(\protect\hyperlink{ref-hofhuis2016DealingDifferencesImpact}{2016})
relate warmth and competence perceptions to social and task-performance
ratings HR managers assigned to job candidates. In textual data
analysis, the SCM's applications have extended into computational
research as a framework for stereotype- and bias-detection natural
language models
(\protect\hyperlink{ref-nicolas2020ComprehensiveStereotypeContent}{Nicolas
et al., 2020}).

The SCM differentiates pancultural stereotype content along two
perceptual dimensions: \emph{warmth} and \emph{competence}
(\protect\hyperlink{ref-cuddy2009StereotypeContentModel}{Cuddy et al.,
2009}). Perceived warmth is related to compassion, kindness,
helpfulness, and interpersonal sensitivity whereas perceived competence
is associated with self-assertion, leadership, analytical thinking, and
independence (for a list of traits, see
\protect\hyperlink{ref-bruckmuller2013DensityBigTwo}{Bruckmüller \&
Abele, 2013}; \protect\hyperlink{ref-Carli2016}{Carli et al., 2016};
\protect\hyperlink{ref-hummert1990MultipleStereotypesElderly}{Hummert,
1990}). The assessments of outgroup members along these two dimensions
form the core of social category stereotypes including ingroup
self-stereotypes
(\protect\hyperlink{ref-hinton2019ExploringRelationshipGay}{Hinton et
al., 2019}).

According to the SCM, gender groups are social categories that are
subject to cross-cultural stereotyping along the dimensions of warmth
and competence (\protect\hyperlink{ref-fiske2002ModelOftenMixed}{Fiske
et al., 2002}). Females (and women generally) are linked to warmth
traits but perceived as low in competence whereas males (and men
generally) are linked to competence traits but perceived as low in
warmth (\protect\hyperlink{ref-Eagly1997}{Eagly, 1997};
\protect\hyperlink{ref-suh2004GenderRelationshipsInfluences}{Suh et al.,
2004}). Different age groups also have associated warmth- and
competence-related stereotypes: older individuals are perceived as
lacking in competence compared to their younger counterparts but
generally rated higher on warmth whereas younger individuals are
perceived as lacking in warmth but consistently rated higher on
competence (\protect\hyperlink{ref-cuddy2005ThisOldStereotype}{Cuddy et
al., 2005}; \protect\hyperlink{ref-vanselm2021SearchOlderWorker}{van
Selm \& van den Heijkant, 2021}).

\hypertarget{gender}{%
\subsubsection{Gender Stereotypes in the Occupational
Domain}\label{gender}}

The stereotypical attribution of warmth and competence to females and
males also form the basis for stereotypes about female and male workers
(\protect\hyperlink{ref-froehlich2020GenderWorkNations}{Froehlich et
al., 2020}) and is further generalizable to gendered occupational
domains. Various studies point to the observability of gendered warmth
and competence stereotype differences in different contexts and when
employing different analytical approaches
(\protect\hyperlink{ref-aaldering2020PoliticalLeadershipMedia}{Aaldering
\& Van Der Pas, 2020}; \protect\hyperlink{ref-harmer_are_2017}{Harmer et
al., 2017}). Smith et al.
(\protect\hyperlink{ref-smith2019PowerLanguageGender}{2019}) found that
positive attribute assignments to female and male leaders were aligned
with the SCM, however, female leaders were also attributed more negative
warmth characteristics compared to their male counterparts. These
findings implicitly provide evidence for social categorization framing.
Dominant social groups with stereotyped characteristics matching (and
seen as essential to) a social role are appraised based on said
characteristics whereas non-dominant groups with mismatching stereotyped
characteristics are appraised via both characteristics of the social
role and the social group -- the latter often resulting in unfavorable
appraisal.

Particular to occupational segregation, a survey by He et al.
(\protect\hyperlink{ref-he2019StereotypesWorkOccupational}{2019}) on
warmth and competence perception associated with different occupations
found a positive correlation between occupational stereotype content and
the respective level of gender segregation. Nursing, medical assistance,
childcare, and secretarial work were the highest-rated occupations on
warmth, and women made up the majority in these occupations: 89.4\%,
90.7\%, 94.9\%, and 94.5\% respectively. Similarly, Strinić et al.
(\protect\hyperlink{ref-strinic2021OccupationalStereotypesProfessionals}{2021}),
in a survey using a sample of 130 HR professionals, found that
stereotypical perceptions of warmth and competence are in fact attached
to occupations and sectors.

In light of the reviewed literature, we expect similar differences in
the presence of warmth- and competence-related stereotypical frames in
job ads at the sectoral level:

\begin{quote}
\textbf{Hypothesis 1:} Job advertisements from female-dominated sectors
will include more warmth-related frames when compared to job
advertisements from male-dominated sectors \textbf{(H1a)} or
mixed-gender sectors \textbf{(H1b)}.
\end{quote}

\begin{quote}
\textbf{Hypothesis 2:} Job advertisements from male-dominated sectors
will include more competence-related frames when compared to job
advertisements from female-dominated sectors \textbf{(H2a)} or
mixed-gender sectors \textbf{(H2b)}.
\end{quote}

\hypertarget{age}{%
\subsubsection{Age Stereotypes in the Occupational Domain}\label{age}}

Stereotypical attribution of warmth and competence to workers from
different age groups also aligns with the general stereotypes of
individuals in those categories. In a frame analysis study of Dutch
media texts published over the span of six years, Kroon et al.
(\protect\hyperlink{ref-kroon2018ReliableUnproductiveStereotypes}{2018})
found that both corporate and news media portray older workers as
trustworthy, involved, and committed (warmth characteristics) but
lacking in aptitudes related to productivity, adaptability, and
technological skills (competence characteristics). Krings et al.
(\protect\hyperlink{ref-krings2011StereotypicalInferencesMediators}{2011})
also found that good-naturedness, amicability, benevolence, and
sincerity formed the content of warmth-related stereotypes for older
workers while capability, efficiency, and skill formed the content of
competence-related stereotypes for younger workers.

In job ads, different contextual factors seem to be at play. van Selm \&
van den Heijkant
(\protect\hyperlink{ref-vanselm2021SearchOlderWorker}{2021}) found hard
abilities requirements for general job seekers (e.g., business
operations, leadership, and professional development abilities) were
more pronounced compared to soft abilities requirements for older
workers (e.g., customer service ability). This emphasis on competence
over warmth was also noted in a study by Abrams et al.
(\protect\hyperlink{ref-abrams2016OldUnemployableHow}{2016}) where role
congruity between a job's age-type and an older candidate's stereotyped
characteristics did not increase older candidate selection. Findings
point to an undervaluing of older workers' warmth characteristics and
indicate warmth-primacy -- wherein proneness to evaluate others
positively is predicted primarily by warmth perceptions
(\protect\hyperlink{ref-cuddy2008WarmthCompetenceUniversal}{Cuddy et
al., 2008};
\protect\hyperlink{ref-ponsi2016InfluenceWarmthCompetence}{Ponsi et al.,
2016}) -- functions differently in the context of age-typed recruitment.

Literature thus suggests that when comparing job ads from
older-worker-dominated and younger-worker-dominated sectors, the
presence of competence-related frames may be more relevant to
determining whether bias against older workers (or in favor of younger
workers) may exist. Notwithstanding, as social stereotypes about older
individuals form the basis for older-worker stereotypes, we expect:

\begin{quote}
\textbf{Hypothesis 3:} Job advertisements from sectors dominated by
older workers will include more warmth-related frames when compared to
job advertisements from sectors dominated by younger workers
\textbf{(H3a)} or mixed-age sectors \textbf{(H3b)}.
\end{quote}

\begin{quote}
\textbf{Hypothesis 4:} Job advertisements from sectors dominated by
younger workers will include more competence-related frames when
compared to job advertisements from sectors dominated by older workers
\textbf{(H4a)} or mixed-age sectors \textbf{(H4b)}.
\end{quote}

\hypertarget{method}{%
\subsection{Methodology}\label{method}}

\hypertarget{sampling}{%
\subsubsection{Data Collection and Sample}\label{sampling}}

Job ads were collected based on searches for sectors keywords from three
online job search platforms: LinkedIn.nl, Indeed.nl, and Glassdoor.nl .
Search keywords for sectors were obtained from the one-digit
International Standard Industrial Classifications
(\protect\hyperlink{ref-centraal_bureau_voor_de_statistiek_standard_2018}{Centraal
Bureau voor de Statistiek, 2018}). Some of the original 29 SBI sector
titles were a combination of multiple independent sector designations,
e.g., ``agriculture, forestry and fishing''. These sector titles
returned imprecise search results and were thus divided into distinct
sector search keywords and the resultant list of sectors keywords was
supplemented with data from the more detailed standardized 5-digit SBI
codes and the Dutch Labour Force Survey
(\protect\hyperlink{ref-centraal_bureau_voor_de_statistiek_dutch_2021}{Centraal
Bureau voor de Statistiek, 2021}). This gave a total of 97
sector-keywords, and all returned job postings were automatically
retrieved using Python 3.9 (see Table 9 in Appendix A).

\hypertarget{references}{%
\subsection*{References}\label{references}}
\addcontentsline{toc}{subsection}{References}

\hypertarget{refs}{}
\begin{CSLReferences}{1}{0}
\leavevmode\vadjust pre{\hypertarget{ref-aaldering2020PoliticalLeadershipMedia}{}}%
Aaldering, L., \& Van Der Pas, D. J. (2020). Political {Leadership} in
the {Media}: {Gender Bias} in {Leader Stereotypes} during {Campaign} and
{Routine Times}. \emph{British Journal of Political Science},
\emph{50}(3), 911--931. \url{https://doi.org/10.1017/S0007123417000795}

\leavevmode\vadjust pre{\hypertarget{ref-abrams2016OldUnemployableHow}{}}%
Abrams, D., Swift, H. J., \& Drury, L. (2016). Old and {Unemployable}?
{How Age-Based Stereotypes Affect Willingness} to {Hire Job Candidates}.
\emph{Journal of Social Issues}, \emph{72}(1), 105--121.
\url{https://doi.org/10.1111/josi.12158}

\leavevmode\vadjust pre{\hypertarget{ref-bastian2006PsychologicalEssentialismStereotype}{}}%
Bastian, B., \& Haslam, N. (2006). Psychological essentialism and
stereotype endorsement. \emph{Journal of Experimental Social
Psychology}, \emph{42}(2), 228--235.
\url{https://doi.org/10.1016/j.jesp.2005.03.003}

\leavevmode\vadjust pre{\hypertarget{ref-beattie2012PossibleUnconsciousBias}{}}%
Beattie, G., \& Johnson, P. (2012). Possible unconscious bias in
recruitment and promotion and the need to promote equality.
\emph{Perspectives: Policy and Practice in Higher Education},
\emph{16}(1), 7--13. \url{https://doi.org/10.1080/13603108.2011.611833}

\leavevmode\vadjust pre{\hypertarget{ref-beukeboom2019HowStereotypesAre}{}}%
Beukeboom, C. J., \& Burgers, C. (2019). How stereotypes are shared
through language: {A} review and introduction of the {Social Categories}
and {Stereotypes Communication} ({SCSC}) framework. \emph{Review of
Communication Research}, \emph{7}(2019), 1--37.
\url{https://doi.org/10.12840/issn.2255-4165.017}

\leavevmode\vadjust pre{\hypertarget{ref-brown2003BlackwellHandbookSocial}{}}%
Brown, R., \& Gaertner, S. L. (Eds.). (2003). \emph{Blackwell {Handbook}
of {Social Psychology}: {Intergroup Processes}}. {Blackwell Publishers
Ltd}. \url{https://doi.org/10.1002/9780470693421}

\leavevmode\vadjust pre{\hypertarget{ref-bruckmuller2013DensityBigTwo}{}}%
Bruckmüller, S., \& Abele, A. (2013). The {Density} of the {Big Two How
Are Agency} and {Communion Structurally Represented}? \emph{Social
Psychology}, \emph{44}(2, 2), 63--74.
\url{https://doi.org/10.1027/1864-9335/a000145}

\leavevmode\vadjust pre{\hypertarget{ref-carducci_twitpersonality_2018}{}}%
Carducci, G., Rizzo, G., Monti, D., Palumbo, E., \& Morisio, M. (2018).
{TwitPersonality}: {Computing Personality Traits} from {Tweets Using
Word Embeddings} and {Supervised Learning}. \emph{Information},
\emph{9}(5, 5), 127. \url{https://doi.org/10.3390/info9050127}

\leavevmode\vadjust pre{\hypertarget{ref-Carli2016}{}}%
Carli, L. L., Alawa, L., Lee, Y. A., Zhao, B., \& Kim, E. (2016).
Stereotypes {About Gender} and {Science}: {Women} ≠ {Scientists}.
\emph{Psychology of Women Quarterly}, \emph{40}(2), 244--260.
\url{https://doi.org/10.1177/0361684315622645}

\leavevmode\vadjust pre{\hypertarget{ref-carrillo-tudela_extent_2016}{}}%
Carrillo-Tudela, C., Hobijn, B., She, P., \& Visschers, L. (2016). The
extent and cyclicality of career changes: {Evidence} for the {U}.{K}.
\emph{European Economic Review}, \emph{84}, 18--41.
\url{https://doi.org/10.1016/j.euroecorev.2015.09.008}

\leavevmode\vadjust pre{\hypertarget{ref-centraal_bureau_voor_de_statistiek_standard_2018}{}}%
Centraal Bureau voor de Statistiek. (2018). \emph{Standard {Industrial
Classifications} ({Dutch SBI} 2008, {NACE} and {ISIC})} {[}Database{]}.
{Centraal Bureau voor de Statistiek (CBS)}.
\url{https://www.cbs.nl/en-gb/onze-diensten/methods/classifications/activiteiten/standard-industrial-classifications--dutch-sbi-2008-nace-and-isic--}

\leavevmode\vadjust pre{\hypertarget{ref-centraal_bureau_voor_de_statistiek_dutch_2021}{}}%
Centraal Bureau voor de Statistiek. (2021). \emph{Dutch {Labour Force
Survey} ({LFS})} {[}Database{]}. {Centraal Bureau voor de Statistiek
(CBS)}.
\url{https://www.cbs.nl/en-gb/onze-diensten/methods/surveys/korte-onderzoeksbeschrijvingen/dutch-labour-force-survey--lfs--}

\leavevmode\vadjust pre{\hypertarget{ref-clarke2020GenderStereotypesGenderTyped}{}}%
Clarke, H. M. (2020). Gender {Stereotypes} and {Gender-Typed Work}. In
K. F. Zimmermann (Ed.), \emph{Handbook of {Labor}, {Human Resources} and
{Population Economics}} (pp. 1--23). {Springer International
Publishing}. \url{https://doi.org/10.1007/978-3-319-57365-6_21-1}

\leavevmode\vadjust pre{\hypertarget{ref-cuddy2008WarmthCompetenceUniversal}{}}%
Cuddy, A. J. C., Fiske, S. T., \& Glick, P. (2008). Warmth and
{Competence} as {Universal Dimensions} of {Social Perception}: {The
Stereotype Content Model} and the {BIAS Map}. In \emph{Advances in
{Experimental Social Psychology}} (Vol. 40, pp. 61--149). {Academic
Press}. \url{https://doi.org/10.1016/S0065-2601(07)00002-0}

\leavevmode\vadjust pre{\hypertarget{ref-cuddy2009StereotypeContentModel}{}}%
Cuddy, A. J. C., Fiske, S. T., Kwan, V. S. Y., Glick, P., Demoulin, S.,
Leyens, J.-P., Bond, M. H., Croizet, J.-C., Ellemers, N., Sleebos, E.,
Htun, T. T., Kim, H.-J., Maio, G., Perry, J., Petkova, K., Todorov, V.,
Rodríguez‐Bailón, R., Morales, E., Moya, M., \ldots{} Ziegler, R.
(2009). Stereotype content model across cultures: {Towards} universal
similarities and some differences. \emph{British Journal of Social
Psychology}, \emph{48}(1), 1--33.
\url{https://doi.org/10.1348/014466608x314935}

\leavevmode\vadjust pre{\hypertarget{ref-cuddy2005ThisOldStereotype}{}}%
Cuddy, A. J. C., Norton, M. I., \& Fiske, S. T. (2005). This {Old
Stereotype}: {The Pervasiveness} and {Persistence} of the {Elderly
Stereotype}. \emph{Journal of Social Issues}, \emph{61}(2), 267--285.
\url{https://doi.org/10.1111/j.1540-4560.2005.00405.x}

\leavevmode\vadjust pre{\hypertarget{ref-decooman2012PortrayingFittingValues}{}}%
De Cooman, R., \& Pepermans, R. (2012). Portraying fitting values in job
advertisements. \emph{Personnel Review}, \emph{41}(2), 216--232.
\url{https://doi.org/10.1108/00483481211200042}

\leavevmode\vadjust pre{\hypertarget{ref-Eagly1997}{}}%
Eagly, A. H. (1997). Sex differences in social behavior: Comparing
social role theory and evolutionary psychology. \emph{The American
Psychologist}, \emph{52}(12), 1380--1383.
\url{https://doi.org/10.1037/0003-066X.52.12.1380.b}

\leavevmode\vadjust pre{\hypertarget{ref-entman1993FramingClarificationFractured}{}}%
Entman, R. M. (1993). Framing: {Toward Clarification} of a {Fractured
Paradigm}. \emph{Journal of Communication}, \emph{43}(4), 51--58.
\url{https://doi.org/10.1111/j.1460-2466.1993.tb01304.x}

\leavevmode\vadjust pre{\hypertarget{ref-fiske2002ModelOftenMixed}{}}%
Fiske, S. T., Cuddy, A. J. C., Glick, P., \& Xu, J. (2002). A model of
(often mixed) stereotype content: {Competence} and warmth respectively
follow from perceived status and competition. \emph{Journal of
Personality and Social Psychology}, \emph{82}(6), 878--902.
\url{https://doi.org/10.1037/0022-3514.82.6.878}

\leavevmode\vadjust pre{\hypertarget{ref-fritsch_horizontal_2020}{}}%
Fritsch, N.-S., Liedl, B., \& Paulinger, G. (2020). Horizontal and
vertical labour market movements in {Austria}: {Do} occupational
transitions take women across gendered lines? \emph{Current Sociology},
0011392120969767. \url{https://doi.org/10.1177/0011392120969767}

\leavevmode\vadjust pre{\hypertarget{ref-froehlich2020GenderWorkNations}{}}%
Froehlich, L., Olsson, M. I. T., Dorrough, A. R., \& Martiny, S. E.
(2020). Gender at {Work Across Nations}: {Men} and {Women Working} in
{Male-Dominated} and {Female-Dominated Occupations} are {Differentially
Associated} with {Agency} and {Communion}. \emph{Journal of Social
Issues}, \emph{76}(3), 484--511.
\url{https://doi.org/10.1111/josi.12390}

\leavevmode\vadjust pre{\hypertarget{ref-garcia-retamero2006PrejudiceWomenMalecongenial}{}}%
Garcia-Retamero, R., \& López-Zafra, E. (2006). Prejudice against
{Women} in {Male-congenial Environments}: {Perceptions} of {Gender Role
Congruity} in {Leadership}. \emph{Sex Roles}, \emph{55}(1), 51--61.
\url{https://doi.org/10.1007/s11199-006-9068-1}

\leavevmode\vadjust pre{\hypertarget{ref-Zanna2013}{}}%
Gardner, R. (1994). Stereotypes as consensual beliefs. In M. P. Zanna \&
J. M. Olson (Eds.), \emph{The {Psychology} of {Prejudice}: {The Ontario
Symposium}, {Volume} 7} (pp. 1--31).
\url{https://doi.org/10.4324/9780203763377}

\leavevmode\vadjust pre{\hypertarget{ref-gaucher2011EvidenceThatGendered}{}}%
Gaucher, D., Friesen, J., \& Kay, A. C. (2011). Evidence {That Gendered
Wording} in {Job Advertisements Exists} and {Sustains Gender
Inequality}. \emph{Journal of Personality and Social Psychology},
\emph{101}(1), 109--128. \url{https://doi.org/10.1037/a0022530}

\leavevmode\vadjust pre{\hypertarget{ref-hardin2002FramingSexualDifference}{}}%
Hardin, M., Lynn, S., Walsdorf, K., \& Hardin, B. (2002). The {Framing}
of {Sexual Difference} in {SI} for {Kids Editorial Photos}. \emph{Mass
Communication and Society}, \emph{5}(3), 341--359.
\url{https://doi.org/10.1207/S15327825MCS0503_6}

\leavevmode\vadjust pre{\hypertarget{ref-harmer_are_2017}{}}%
Harmer, E., Savigny, H., \& Ward, O. (2017). {``{Are} you tough
enough?''} {Performing} gender in the {UK} leadership debates 2015.
\emph{Media, Culture \& Society}, \emph{39}(7), 960--975.
\url{https://doi.org/10.1177/0163443716682074}

\leavevmode\vadjust pre{\hypertarget{ref-he2019StereotypesWorkOccupational}{}}%
He, J. C., Kang, S. K., Tse, K., \& Toh, S. M. (2019). Stereotypes at
work: {Occupational} stereotypes predict race and gender segregation in
the workforce. \emph{Journal of Vocational Behavior}, \emph{115},
103318. \url{https://doi.org/10.1016/j.jvb.2019.103318}

\leavevmode\vadjust pre{\hypertarget{ref-heilman2012GenderStereotypesWorkplace}{}}%
Heilman, M. E. (2012). Gender stereotypes and workplace bias.
\emph{Research in Organizational Behavior}, \emph{32}, 113--135.
\url{https://doi.org/10.1016/j.riob.2012.11.003}

\leavevmode\vadjust pre{\hypertarget{ref-hinton2019ExploringRelationshipGay}{}}%
Hinton, J. D. X., Anderson, J. R., \& Koc, Y. (2019). Exploring the
relationship between gay men's self- and meta-stereotype endorsement
with well-being and self-worth. \emph{Psychology \& Sexuality},
\emph{10}(2), 169--182.
\url{https://doi.org/10.1080/19419899.2019.1577013}

\leavevmode\vadjust pre{\hypertarget{ref-hofhuis2016DealingDifferencesImpact}{}}%
Hofhuis, J., van der Zee, K. I., \& Otten, S. (2016). Dealing with
differences: The impact of perceived diversity outcomes on selection and
assessment of minority candidates. \emph{The International Journal of
Human Resource Management}, \emph{27}(12), 1319--1339.
\url{https://doi.org/10.1080/09585192.2015.1072100}

\leavevmode\vadjust pre{\hypertarget{ref-hummert1990MultipleStereotypesElderly}{}}%
Hummert, M. L. (1990). Multiple stereotypes of elderly and young adults:
{A} comparison of structure and evaluations. \emph{Psychology and
Aging}, \emph{5}(2), 182--193.
\url{https://doi.org/10.1037/0882-7974.5.2.182}

\leavevmode\vadjust pre{\hypertarget{ref-kelly2010GenderedChallengeGendered}{}}%
Kelly, E. L., Ammons, S. K., Chermack, K., \& Moen, P. (2010). Gendered
{Challenge}, {Gendered Response}: {Confronting} the {Ideal Worker Norm}
in a {White-Collar Organization}. \emph{Gender \& Society},
\emph{24}(3), 281--303. \url{https://doi.org/10.1177/0891243210372073}

\leavevmode\vadjust pre{\hypertarget{ref-krings2011StereotypicalInferencesMediators}{}}%
Krings, F., Sczesny, S., \& Kluge, A. (2011). Stereotypical {Inferences}
as {Mediators} of {Age Discrimination}: {The Role} of {Competence} and
{Warmth}. \emph{British Journal of Management}, \emph{22}(2), 187--201.
\url{https://doi.org/10.1111/j.1467-8551.2010.00721.x}

\leavevmode\vadjust pre{\hypertarget{ref-kroon2018ReliableUnproductiveStereotypes}{}}%
Kroon, A. C., Van Selm, M., ter HOEVEN, C. L., \& Vliegenthart, R.
(2018). Reliable and unproductive? {Stereotypes} of older employees in
corporate and news media. \emph{Ageing and Society}, \emph{38}(1),
166--191. \url{https://doi.org/10.1017/S0144686X16000982}

\leavevmode\vadjust pre{\hypertarget{ref-meeks_all_2013}{}}%
Meeks, L. (2013). All the {Gender That}'s {Fit} to {Print}: {How} the
{New York Times Covered Hillary Clinton} and {Sarah Palin} in 2008.
\emph{Journalism \& Mass Communication Quarterly}, \emph{90}(3),
520--539. \url{https://doi.org/10.1177/1077699013493791}

\leavevmode\vadjust pre{\hypertarget{ref-nicolas2020ComprehensiveStereotypeContent}{}}%
Nicolas, G., Bai, X., \& Fiske, S. T. (2020). Comprehensive {Stereotype
Content Dictionaries Using} a {Semi}‐{Automated Method}. \emph{European
Journal of Social Psychology}.

\leavevmode\vadjust pre{\hypertarget{ref-paleari2019WhenPrejudiceYou}{}}%
Paleari, F. G., Brambilla, M., \& Fincham, F. D. (2019). When prejudice
against you hurts others and me: {The} case of ageism at work.
\emph{Journal of Applied Social Psychology}, \emph{49}(11), 704--720.
\url{https://doi.org/10.1111/jasp.12628}

\leavevmode\vadjust pre{\hypertarget{ref-ponsi2016InfluenceWarmthCompetence}{}}%
Ponsi, G., Panasiti, M. S., Scandola, M., \& Aglioti, S. M. (2016).
Influence of warmth and competence on the promotion of safe in-group
selection: {Stereotype} content model and social categorization of
faces. \emph{Quarterly Journal of Experimental Psychology},
\emph{69}(8), 1464--1479.
\url{https://doi.org/10.1080/17470218.2015.1084339}

\leavevmode\vadjust pre{\hypertarget{ref-reese2001FramingPublicLife}{}}%
Reese, S. D., Jr, O. H. G., \& Grant, A. E. (2001). \emph{Framing
{Public Life}: {Perspectives} on {Media} and {Our Understanding} of the
{Social World}}. {Routledge}.
\url{https://books.google.com?id=LhaQAgAAQBAJ}

\leavevmode\vadjust pre{\hypertarget{ref-RynesS.1989}{}}%
Rynes, S. L. (1989). Recruitment, job choice, and post-hire
consequences: {A} call for new research directions. In M. D. Dunnette \&
L. M. Hough (Eds.), \emph{Handbook of {Industrial} and {Organizational
Psychology}} (Vol. 2, pp. 399--444). {Consulting Psychologists Press}.
\url{https://psycnet.apa.org/record/1993-97200-007}

\leavevmode\vadjust pre{\hypertarget{ref-smith2019PowerLanguageGender}{}}%
Smith, D. G., Rosenstein, J. E., Nikolov, M. C., \& Chaney, D. A.
(2019). The {Power} of {Language}: {Gender}, {Status}, and {Agency} in
{Performance Evaluations}. \emph{Sex Roles}, \emph{80}(3), 159--171.
\url{https://doi.org/10.1007/s11199-018-0923-7}

\leavevmode\vadjust pre{\hypertarget{ref-strinic2021OccupationalStereotypesProfessionals}{}}%
Strinić, A., Carlsson, M., \& Agerström, J. (2021). Occupational
stereotypes: Professionals´ warmth and competence perceptions of
occupations. \emph{Personnel Review}, \emph{ahead-of-print}.
\url{https://doi.org/10.1108/PR-06-2020-0458}

\leavevmode\vadjust pre{\hypertarget{ref-suh2004GenderRelationshipsInfluences}{}}%
Suh, E. J., Moskowitz, D. S., Fournier, M. A., \& Zuroff, D. C. (2004).
Gender and relationships: {Influences} on agentic and communal
behaviors. \emph{Personal Relationships}, \emph{11}(1), 41--60.
\url{https://doi.org/10.1111/j.1475-6811.2004.00070.x}

\leavevmode\vadjust pre{\hypertarget{ref-sullivan19891984VicePresidential}{}}%
Sullivan, P. A. (1989). The 1984 vice‐presidential debate: {A} case
study of female and male framing in political campaigns.
\emph{Communication Quarterly}, \emph{37}(4), 329--343.
\url{https://doi.org/10.1080/01463378909385554}

\leavevmode\vadjust pre{\hypertarget{ref-vangorp2005WhereFrameVictims}{}}%
Van Gorp, B. (2005). Where is the {Frame}?: {Victims} and {Intruders} in
the {Belgian Press Coverage} of the {Asylum Issue}. \emph{European
Journal of Communication}, \emph{20}(4), 484--507.
\url{https://doi.org/10.1177/0267323105058253}

\leavevmode\vadjust pre{\hypertarget{ref-vanselm2021SearchOlderWorker}{}}%
van Selm, M., \& van den Heijkant, L. (2021). In {Search} of the {Older
Worker}: {Framing Job Requirements} in {Recruitment Advertisements}.
\emph{Work, Aging and Retirement}, \emph{7}(4), 288--302.
\url{https://doi.org/10.1093/workar/waaa026}

\leavevmode\vadjust pre{\hypertarget{ref-westerhof2010FillingMissingLink}{}}%
Westerhof, G. J., Harink, K., Selm, M. V., Strick, M., \& Baaren, R. V.
(2010). Filling a missing link: The influence of portrayals of older
characters in television commercials on the memory performance of older
adults. \emph{Ageing \& Society}, \emph{30}(5), 897--912.
\url{https://doi.org/10.1017/S0144686X10000152}

\leavevmode\vadjust pre{\hypertarget{ref-white2009ThinkWomenThink}{}}%
White, J. B., \& Gardner, W. L. (2009). Think {Women}, {Think Warm}:
{Stereotype Content Activation} in {Women} with a {Salient Gender
Identity}, {Using} a {Modified Stroop Task}. \emph{Sex Roles},
\emph{60}(3-4), 247--260.
\url{https://doi.org/10.1007/s11199-008-9526-z}

\leavevmode\vadjust pre{\hypertarget{ref-Yang2015a}{}}%
Yang, A. (2015). Building a {Cognitive-Sociological Model} of
{Stereotypes}: {Stereotypical Frames}, {Social Distance} and {Framing
Effects}. \emph{Howard Journal of Communications}, \emph{26}(3),
254--274. \url{https://doi.org/10.1080/10646175.2015.1049757}

\end{CSLReferences}



\end{document}
